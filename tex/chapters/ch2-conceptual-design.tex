\chapter{Conceptual Design}\label{ch:2}
To build the Dimensional Fact Model, we first think about which dimensions and measures our fact should have. Since we are describing the Product Usage and Engagement our fact will represent a single atomic user event in Notion such as these described in Chapter \ref{ch:1}. To capture the full behavioral context, we need to capture also the user, the content object, the workspace, the device, the event type, the session and the timestamp, which we do via dimensions. We therefore have the following dimensions:
\begin{table}[H]
\centering
\begin{tabular}{ll}
\hline
\textbf{Attribute} & \textbf{Description} \\
\hline
\texttt{user\_id} & Unique identifier of the user \\
\texttt{signup\_date} & Date the user registered \\
\texttt{subscription\_tier} & Free, Plus, Business, Enterprise \\
\texttt{user\_type} & Individual, member, guest \\
\texttt{region} & User's geographic region \\
\texttt{lifecycle\_stage} & Onboarding, active, dormant, churn-risk \\
\hline
\end{tabular}
\caption{User Dimension}
\end{table}


\begin{table}[H]
\centering
\begin{tabular}{ll}
\hline
\textbf{Attribute} & \textbf{Description} \\
\hline
\texttt{content\_id} & Identifier of the page or block \\
\texttt{content\_type} & Page, database, wiki, task board, etc. \\
\texttt{is\_template\_based} & Indicates template-derived content \\
\texttt{is\_shared} & Whether content is shared with others \\
\texttt{ownership\_type} & Personal, workspace, or team space \\
\hline
\end{tabular}
\caption{Content Dimension}
\end{table}


\begin{table}[H]
\centering
\begin{tabular}{ll}
\hline
\textbf{Attribute} & \textbf{Description} \\
\hline
\texttt{workspace\_id} & Unique identifier of the workspace \\
\texttt{workspace\_plan} & Subscription plan of the workspace \\
\texttt{workspace\_size\_bucket} & Size category (1, 2--10, 11--50, 50+) \\
\texttt{industry\_segment} & Workspace industry classification \\
\texttt{workspace\_region} & Geographic region of workspace \\
\hline
\end{tabular}
\caption{Workspace Dimension}
\end{table}


\begin{table}[H]
\centering
\begin{tabular}{ll}
\hline
\textbf{Attribute} & \textbf{Description} \\
\hline
\texttt{device\_id} & Identifier for the device type \\
\texttt{platform} & Web, desktop, mobile \\
\texttt{operating\_system} & OS used for the event \\
\texttt{app\_version} & Client version \\
\texttt{device\_form\_factor} & Phone, tablet, desktop \\
\hline
\end{tabular}
\caption{Device Dimension}
\end{table}


\begin{table}[H]
\centering
\begin{tabular}{ll}
\hline
\textbf{Attribute} & \textbf{Description} \\
\hline
\texttt{event\_type} & View, edit, create, comment, share, etc. \\
\texttt{feature\_category} & Databases, templates, integrations, collaboration \\
\texttt{interaction\_intent} & Consumption, creation, collaboration \\
\hline
\end{tabular}
\caption{Event Dimension}
\end{table}


\begin{table}[H]
\centering
\begin{tabular}{ll}
\hline
\textbf{Attribute} & \textbf{Description} \\
\hline
\texttt{session\_id} & Unique session identifier \\
\texttt{session\_start\_time} & Start timestamp of the session \\
\texttt{session\_end\_time} & End timestamp of the session \\
\texttt{session\_origin} & Entry source (direct, link, notification) \\
\hline
\end{tabular}
\caption{Session Dimension}
\end{table}


\begin{table}[H]
\centering
\begin{tabular}{ll}
\hline
\textbf{Attribute} & \textbf{Description} \\
\hline
\texttt{date\_key} & Surrogate key for the date \\
\texttt{calendar\_date} & Full calendar date \\
\texttt{day\_of\_week} & Numeric day of week \\
\texttt{is\_weekend} & Weekend indicator \\
\texttt{week\_of\_year} & ISO week number \\
\texttt{month} & Month number \\
\texttt{quarter} & Calendar quarter \\
\texttt{year} & Calendar year \\
\texttt{day\_since\_signup\_bucket} & Bucket for activation/cohort logic \\
\hline
\end{tabular}
\caption{Time Dimension}
\end{table}

To evaluate product usage and engagement effectively, the fact table must also include measures that quantify how users interact with Notion. Therefore, the following measures allow us to aggregate individual events into meaningful indicators:

\textbf{Event Count}
\begin{itemize} % Important for possible aggregations later on
    \item \textbf{Description:} Always 1 per event.
    \item \textbf{Additivity:} Fully additive across all dimensions.
\end{itemize}

\textbf{Active User Flag}
\begin{itemize}
    \item \textbf{Description:} 1 on a user's first event of a day, 0 otherwise.
    \item \textbf{Additivity:} Additive across users for a fixed day; semi-additive across time.
\end{itemize}

\textbf{Activation Event Flag}
\begin{itemize}
    \item \textbf{Description:} 1 if the event is a meaningful interaction within the first seven days after signup.
    \item \textbf{Additivity:} Additive within user-time windows; non-additive when converted into activation rates.
\end{itemize}

\textbf{Feature Usage Flag}
\begin{itemize}
    \item \textbf{Description:} 1 if the event uses an advanced feature.
    \item \textbf{Additivity:} Additive for counts; non-additive when used in ratios.
\end{itemize}

\textbf{Collaboration Event Flag}
\begin{itemize}
    \item \textbf{Description:} 1 for events occurring in a collaborative context.
    \item \textbf{Additivity:} Additive for counts; non-additive for proportions.
\end{itemize}

\textbf{Session Event Count}
\begin{itemize}
    \item \textbf{Description:} Number of events per session (sum of Event Count grouped by session).
    \item \textbf{Additivity:} Additive within a session; semi-additive across time.
\end{itemize}

\textbf{Session Duration Seconds}
\begin{itemize}
    \item \textbf{Description:} Stored on the final event of a session.
    \item \textbf{Additivity:} Additive across sessions; semi-additive across time.
\end{itemize}


\section{Dimensional Fact Model}
Since we now discussed and defined the dimensions and measures of our Product Usage and Engagement fact, we can start building the Dimensional Fact Model (DFM).

\begin{figure}[ht]
\centering
\resizebox{\linewidth}{!}{%
\begin{tikzpicture}[
  font=\sffamily\footnotesize,
  fact/.style={draw, rounded corners, thick, minimum width=40mm, minimum height=18mm, align=center},
  attr/.style={draw, circle, minimum size=2.5mm, inner sep=0pt},
  dimlbl/.style={fill=white, inner sep=1pt},
  >=latex
]

%==================== FACT ====================
\node[fact, draw, rounded corners=0, minimum width=0pt, inner sep=3pt] (F) {
    \begin{tabular}{l}
        \textbf{Product Usage \& Engagement} \\ \hline
        Event\_count \\
        Active\_user\_flag \\
        Activation\_event\_flag \\
        Feature\_usage\_flag \\
        Collaboration\_event\_flag \\
        Session\_event\_count \\
        Session\_duration\_sec \\
    \end{tabular}
};

%==================== TIME DIM ====================
\node[attr,left=26mm of F] (time0) {};
\node[dimlbl,above=2mm of time0] {Time};
\draw[-,thick] (time0) -- (F.west);

% Hierarchy: Date -> Week -> Month -> Quarter -> Year
\node[attr,left=7mm of time0] (timeMonth) {};
\node[below=0mm of timeMonth] {Month};
\node[attr,left=7mm of timeMonth] (timeQuarter) {};
\node[above=0mm of timeQuarter] {Quarter};
\node[attr,left=7mm of timeQuarter] (timeYear) {};
\node[left=0mm of timeYear] {Year};

\draw[thick] (timeYear) -- (timeQuarter) -- (timeMonth) -- (time0);

% Attributes attached to the Date level
\node[attr,above right=2mm and 15mm of time0] (timeDoW) {};
\node[above=0mm of timeDoW] {Week};
\draw[thick] (time0) -- (timeDoW);

\node[attr,above right=10mm and 12mm of time0] (week) {};
\node[above=0mm of week] {Day of week};
\draw[thick] (time0) -- (week);

\node[attr,below right=5mm and 2mm of time0] (timeWeekend) {};
\node[below=0mm of timeWeekend] {Weekend};
\draw[thick] (time0) -- (timeWeekend);

\node[attr,above left=6mm and 7mm of time0] (timeCalDate) {};
\node[above=2mm of timeCalDate] {Calendar date};
\draw[thick] (time0) -- (timeCalDate);

\node[attr,below left=8mm and 4mm of time0] (timeSignupBucket) {};
\node[below left=0mm of timeSignupBucket] {Days s. signup};
\draw[thick] (time0) -- (timeSignupBucket);

\node[attr,below right=3mm and 7mm of time0] (timeKey) {};
\node[right=0mm of timeKey] {Date key};
\draw[thick] (time0) -- (timeKey);

%==================== USER DIM ==================== USER name
\node[attr,above=18mm of F] (user0) {};
\node[dimlbl,above=0mm of user0] {User};
\draw[thick,-] (user0) -- (F.north);

% signup_date
\node[attr,above right=4mm and 10mm of user0] (userSignup) {};
\node[right=0mm of userSignup] {signup\_date};
\draw[thick] (user0) -- (userSignup);

% subscription_tier
\node[attr,right=12mm of user0] (userTier) {};
\node[right=0mm of userTier] {subscription\_tier};
\draw[thick] (user0) -- (userTier);

% user_type
\node[attr,below right=4mm and 10mm of user0] (userType) {};
\node[right=0mm of userType] {user\_type};
\draw[thick] (user0) -- (userType);

% username
\node[inner sep=0pt, minimum size=0pt, below left=4mm and 10mm of user0] (usernameHub) {};
\draw[thick] (user0) -- (usernameHub);

% left branch: username
\coordinate (usernameLeft) at ($(usernameHub)+(-15mm,0mm)$);
\draw[thick] ($(usernameHub)+(.1mm,0mm)$) -- (usernameLeft);
\node[above right=0mm and 0mm of usernameLeft] {username};


% region
\node[attr,left=12mm of user0] (userRegion) {};
\node[left=0mm of userRegion] {region};
\draw[thick] (user0) -- (userRegion);

% lifecycle_stage
\node[attr,above left=4mm and 10mm of user0] (userLife) {};
\node[left=0mm of userLife] {lifecycle\_stage};
\draw[thick] (user0) -- (userLife);


%==================== WORKSPACE DIM ====================
\node[attr,right=26mm of F] (ws0) {};
\node[dimlbl,above=1mm of ws0] {Workspace};
\draw[thick,-] (ws0) -- (F.east);

% subscription plan
\node[attr,below right=1mm and 8mm of ws0] (wsPlan) {};
\node[right=0mm of wsPlan] {workspace\_plan};
\draw[thick] (ws0) -- (wsPlan);

% size bucket
\node[attr,below=5mm of wsPlan] (wsSize) {};
\node[right=0mm of wsSize] {workspace\_size\_bucket};
\draw[thick] (ws0) -- (wsSize);

% industry segment (optional)
\node[attr,below left=5mm and 2mm of ws0] (wsInd) {};
\node[below left=0mm and -10mm of wsInd] {industry\_segment};
\draw[thick] (ws0) -- (wsInd);
% connection with optionality marker
\draw[thick] (ws0) -- (wsInd);
\path (ws0) -- (wsInd) coordinate[pos=0.3] (wsIndOpt);
\draw[thick] ($(wsIndOpt)+(0,-2.2pt)$) -- ($(wsIndOpt)+(0,2.2pt)$);

% geographic region
\node[attr,above right=1mm and 8mm of ws0] (wsRegion) {};
\node[right=0mm of wsRegion] {workspace\_region};
\draw[thick] (ws0) -- (wsRegion);

%==================== CONTENT DIM ==================== content name
\node[attr,below left=16mm and 22mm of F] (cont0) {};
\node[dimlbl,above left=2mm and -4mm of cont0] {Content};
\draw[thick,-] (cont0) -- (F.south west);

% content_type
\node[attr,left=10mm of cont0] (contType) {};
\node[left=0mm of contType] {content\_type};
\draw[thick] (cont0) -- (contType);

% is_template_based
\node[attr,below left=4mm and 10mm of cont0] (contTemplate) {};
\node[left=0mm of contTemplate] {is\_template\_based};
\draw[thick] (cont0) -- (contTemplate);

% is_shared
\node[attr,below=5mm of cont0] (contShared) {};
\node[below=0mm of contShared] {is\_shared};
\draw[thick] (cont0) -- (contShared);

% ownership_type
\node[attr,below right=4mm and 10mm of cont0] (contOwner) {};
\node[right=0mm of contOwner] {ownership\_type};
\draw[thick] (cont0) -- (contOwner);

%==================== DEVICE DIM ====================
\node[attr,below right=16mm and 22mm of F] (dev0) {};
\node[dimlbl,above=2mm of dev0] {Device};
\draw[thick,-] (dev0) -- (F.south east);

% platform
\node[attr,right=12mm of dev0] (devPlatform) {};
\node[right=0mm of devPlatform] {platform};
\draw[thick] (dev0) -- (devPlatform);

% operating_system
\node[attr,below right=4mm and 10mm of dev0] (devOS) {};
\node[right=0mm of devOS] {operating\_system};
\draw[thick] (dev0) -- (devOS);

% app_version
\node[attr,below=5mm of dev0] (devVersion) {};
\node[below=0mm of devVersion] {app\_version};
\draw[thick] (dev0) -- (devVersion);

% device_form_factor
\node[attr,below left=4mm and 10mm of dev0] (devForm) {};
\node[left=0mm of devForm] {device\_form\_factor};
\draw[thick] (dev0) -- (devForm);

%==================== EVENT DIM ====================
\node[attr,above right=16mm and 22mm of F] (evt0) {};
\node[dimlbl,above=1mm of evt0] {Event};
\draw[thick,-] (evt0) -- (F.north east);

% event_type
\node[attr,above right=4mm and 10mm of evt0] (evtType) {};
\node[right=0mm of evtType] {event\_type};
\draw[thick] (evt0) -- (evtType);

% feature_category
\node[attr,right=12mm of evt0] (evtFeat) {};
\node[right=0mm of evtFeat] {feature\_category};
\draw[thick] (evt0) -- (evtFeat);

% interaction_intent
\node[attr,below right=4mm and 10mm of evt0] (evtIntent) {};
\node[right=0mm of evtIntent] {interaction\_intent};
\draw[thick] (evt0) -- (evtIntent);


%==================== SESSION DIM ====================
\node[attr,above left=16mm and 22mm of F] (sess0) {};
\node[dimlbl,above=2mm of sess0] {Session};
\draw[thick,-] (sess0) -- (F.north west);

% session_start_time
\node[attr,above left=4mm and 10mm of sess0] (sessStart) {};
\node[left=0mm of sessStart] {session\_start\_time};
\draw[thick] (sess0) -- (sessStart);

% session_end_time
\node[attr,below left=0mm and 10mm of sess0] (sessEnd) {};
\node[left=0mm of sessEnd] {session\_end\_time};
\draw[thick] (sess0) -- (sessEnd);

% session_origin
\node[attr,below=5mm of sess0] (sessOrigin) {};
\node[below=0mm of sessOrigin] {session\_origin};
\draw[thick] (sess0) -- (sessOrigin);

\end{tikzpicture}
}
\caption{Dimensional Fact Model for Product Usage and Engagement.}
\end{figure}

TODO: Explaining the DFM and why I choose some of the design choices I made

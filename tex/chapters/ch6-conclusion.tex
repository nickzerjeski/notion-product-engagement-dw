\chapter{Conclusion}
After analyzing the domain of our case study in notion, we have identified several key analytical questions that will generate valuable insights. These questions are designed to explore various aspects of the data and provide a comprehensive understanding of the underlying patterns and trends in the company.

After that, we formulated a robust dimensional fact model that effectively captures the relationships and attributes necessary to address these questions. The data model serves as a blueprint for organizing and structuring the data in a way that facilitates efficient analysis.

From the dimensional fact model, we derived a detailed logical schema that outlines the specific tables, fields, and relationships required to implement the data model in a database system. This schema provides a clear roadmap for the physical implementation of the data warehouse.

Next, we developed a comprehensive ETL (Extract, Transform, Load) process to populate the data warehouse with relevant data from various sources. The ETL process ensures that the data is accurately extracted, transformed into the appropriate format, and loaded into the data warehouse for analysis.

Finally, we implemented a series of analytical queries to extract meaningful insights from the data warehouse. These queries are designed to answer the key analytical questions identified earlier and provide actionable information to support decision-making within the company.

In conclusion, the systematic approach we followed—from identifying analytical questions to implementing ETL processes and analytical queries—has laid a solid foundation for effective data analysis in our case study. This structured methodology not only enhances our understanding of the data but also equips us with the tools necessary to derive actionable insights that can drive informed business decisions.